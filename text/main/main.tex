\documentclass[12pt]{article}
\usepackage{amsmath,amssymb,amsthm}
\usepackage{graphicx}
\usepackage{subcaption}
\usepackage{float}
\PassOptionsToPackage{hyphens}{url}\usepackage{hyperref}\usepackage{color}
\usepackage{enumitem}
\usepackage{tabularx}
\usepackage[sorting=none]{biblatex}
\usepackage[raggedrightboxes]{ragged2e}
\setlength{\headheight}{15pt}
\usepackage{fancyhdr}

\usepackage{lineno}
\linenumbers

\usepackage[table,xcdraw]{xcolor}
\usepackage[normalem]{ulem}
\useunder{\uline}{\ul}{}


% figure and bibliography source
\addbibresource{../prime-editing.bib}
% add figure path
\graphicspath{ {../figures/} }
% reduce margin
\usepackage[margin=1in]{geometry}
% increase title font size
\usepackage{titling}
\pretitle{\begin{center}\Huge\bfseries}

\pagestyle{fancy}

\setlist[enumerate]{itemsep=0mm}

\title{An Interactive Review of In-silico Prime Editing Guide Design Tools}
% \author{Peiheng Lu}

\begin{document}
% no date
\date{}
\maketitle

\begin{abstract}
    Prime editing is a novel genome editing technology that enables precise base editing without the need for double-strand breaks. The design of prime editing guides is a critical step in the prime editing workflow. In this review, we evaluate the performance of several in-silico prime editing guide design tools. We compare the quality of the guide designed by these tools, and to improve the usage of these state of art tools, they were reimplemented and integrated into a single web base application. Additionally, we provided the ability to aggregate the results from multiple tools using ensemble learning to improve the overall guide design quality.
    Thus, with the on and off target activity of prime editing quantified, we can provide a complete overview of the outcome of using a specific pegRNA sequence on a specific target loci in a specific cell line. This should noticeably improve the safety and efficiency of prime editing, and thus accelerate its clinical application.
    Thus, with the on and off target activity of prime editing quantified, we can provide a complete overview of the outcome of using a specific pegRNA sequence on a specific target loci in a specific cell line. This should noticeably improve the safety and efficiency of prime editing, and thus accelerate its clinical application.

    \textbf{Keywords:} Prime editing, Machine Learning, in-silico tools, Ensemble Learning
\end{abstract}

\newpage

\section{Background}

Prime editing is a versatile and precise genome editing technology that enables the introduction of all 12 possible base-to-base conversions as well as insertions and deletions without the need for double-strand breaks\cite{liudavidr.SearchreplaceGenomeEditing2019}. 

The versatility of prime editors comes from the fusion of a reverse transcriptase (RT) and a Cas9 nickase (nCas9) to a prime editing guide RNA (pegRNA) (). After the guide RNA binds to the protospacer, the nCas9 creates a single-strand break in the complementary strand, which allows the RT to copy the edited sequence from the pegRNA into the target DNA. This mechanism enables theoretically any types of edits, as RT can be an arbitrary sequence of nucleotides\cite{liudavidr.SearchreplaceGenomeEditing2019}. 

More than 6,000 disorders are known to be caused by various types of mutations in the genome, with around 300 new genetic disorders being discovered each year\cite{petraityteGenomeEditingMedicine2021}. Up to 90\% of these disorder-inducing mutations can be corrected using prime editing\cite{kantorCRISPRCas9DNABaseEditing2020}. However, its clinical application is significantly limited by its relative low editing efficiency at certain target loci.  Empirical methods could be used to identify prime editing guides with high editing efficiency, but they are time-consuming and expensive. Therefore, in-silico prediction tools have garnered significant interest in the scientific community.



\subsection*{In-silico Prime Editing Guide Design Tools}

A number of in silico on target prediction tools have been developed to predict the efficiency of prime editing guides. 

PRIDICT 2 makes a further step towards improving the prediction accuracy by updating the data preprocessing and model training step. By implementing multitask learning sharing the embedding and bidirectional RNN layers, PRIDICT 2 is able to predict the editing efficiency of prime editing guides with higher accuracy than its predecessor\cite{mathisMachineLearningPrediction2024}. 

\begin{figure}
    
\end{figure}
\section{Methods}

\subsection*{Data Acquisition and Preprocessing}

The dataset used in this study was obtained from the DeepPrime and PRIDICT study\cite{mathisMachineLearningPrediction2024,yuPredictionEfficienciesDiverse2023}, which contains 220,000 and 20,000 prime editing guides, respectively. 

\subsection*{Ensemble Learning}

Three ensemble learning approaches were investigated in this study: weighted average, bagging and AdaBoost. The algorithms were implemented in Python, but without the use of Scikit-learn ensemble library, as it does not support having different types of base learners in the ensemble.  

Details of the implementation can be found in \autoref{appendix:ensemble}.

However, no significant difference in performance was observed among the three ensemble learning methods, possibly due to the high correlation in error between the base models (Add figure here). The weighted average method was chosen for the final implementation due to its simplicity and ease of interpretation.
\section{Results}
\section{Discussion}

\newpage
\appendix

\section{Appendix}

\subsection{Ensemble Learning Methods}
\label{appendix:ensemble}



\newpage
\printbibliography
\end{document}